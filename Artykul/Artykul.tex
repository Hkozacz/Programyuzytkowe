\documentclass[a4paper,12pt]{article}
\usepackage[MeX]{polski}
\usepackage[utf8]{inputenc}

%opening
\title{Wydział Matematyki i Informatyki UWM}

\author{}

\begin{document}
\maketitle
Wydział Matematyki i Informatyki Uniwersytetu Warmińsko-Mazurskiego (WMiI) – wydział
Uniwersytetu Warmińsko-Mazurskiego w Olsztynie oferujący studia na dwóch kierunkach:
\begin{itemize}
  \item Informatyka
  \item Matematyka
   \ldots
\end{itemize}
w trybie studiów stacjonarnych i niestacjonarnych. Ponadto oferuje studia podyplomowe.
Wydział zatrudnia 8 profesorów, 14 doktorów habilitowanych, 53 doktorów i 28 magistrów.
\tableofcontents
\section[Misja] {Pierwszy rozdział}
Misją Wydziału jest:
\begin{itemize}
  \item Kształcenie matematyków zdolnych do udziału w rozwijaniu matematyki i jej stosowania w innych
działach wiedzy i w praktyce;

  \item Kształcenie nauczycieli matematyki, nauczycieli matematyki z fizyką a także nauczycieli informatyki;
	\item Kształcenie profesjonalnych informatyków dla potrzeb gospodarki, administracji, szkolnictwa oraz życia
społecznego;
\item Nauczanie matematyki i jej działów specjalnych jak statystyka matematyczna, ekonometria,
biomatematyka, ekologia matematyczna, metody numeryczne; fizyki a w razie potrzeby i podstaw
informatyki na wszystkich wydziałach UWM.
   \ldots
\end{itemize}

\section[Opis kierunków[1]]{Drugi rozdział}
Treść drugie rozdziału.
\end{document}

\begin{abstract}
Wydział Matematyki i Informatyki UWM
\end{abstract}
WMiI liczy ...
\section{}

\end{document}