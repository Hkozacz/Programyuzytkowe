\documentclass[landscape]{slides}
\usepackage[landscape]{geometry}
\usepackage[utf8]{inputenc}
\usepackage{color}
\usepackage{amsfonts}
\usepackage[MeX]{polski}
\usepackage{graphicx}
\begin{document}
\begin{slide}
\textcolor{blue}{\Large{Zawada (wojewodztwo podkarpackie)}\footnote{zrodlo: Wikipedia}}
\begin{itemize}
\item{ wies w Polsce położona w wojewodztwie podkarpackim, w powiecie debickim, w gminie Debica. W latach 1975–1998 miejscowosc nalezała administracyjnie do wojewodztwa tarnowskiego.

Wies polozona poludnikowo, na progu Pogorza Strzyzowskiego i zapadliska przedkarpackiego, wzdłuz rzeki Zawadki (Iselki), przecieta droga krajowa nr 94 i linia kolejowa nr 91.}
\end{itemize}
\end{slide}
\begin{slide} 
\begin{figure}
  \centering
    \includegraphics{pics/zawada_06.jpg}
\end{figure}
\end{slide}
\begin {slide}
\textcolor{blue} {\large {Historia}\footnote{zrodlo: Wikipedia}}
Wieś istniała od XIII wieku. Pierwsza wzmianka o niej pochodzi z 1337 roku. Nazwę miejscowości w zlatynizowanej staropolskiej formie Zawada wymienia w latach (1470-1480) Jan Długosz w księdze Liber beneficiorum dioecesis Cracoviensis.[2] Długosz wymienia również właściciela miejscowości Spytka z Melsztyna.
\end{slide}
\Begine{slide}
\textcolor{blue} {\large {zabytki}\footnote{zrodlo: Wikipedia}}
\begin{itemize}
\item kościół parafialny pw. Narodzenia NMP, 1646, XIX, nr rej.: A-1138 z 30.05.1987
\item zespół dworski, nr rej.: A-253 z 18.03.1972:
\begin{itemize}
\item pozostałości dworu (baszta), XVII i XIX w.
\item oficyna, pocz. XIX w.
\item pałac, 1918
\item pralnia z bramką, pocz. XIX w.
\end{itemize}
\end{itemiza}
\end{slide}
\begin{thebibliography}{9}

\bibitem{lamport94}}.
 GUS. Bank Danych Lokalnych
Joannis Długosz Senioris Canonici Cracoviensis, "Liber Beneficiorum", Aleksander Przezdziecki, Tom II, Kraków 1864, str.290.

\end{thebibliography}
\end{document}
